\documentclass[12pt]{article}
\usepackage[utf8]{inputenc}
\usepackage{amsmath, amssymb} % Math packages
\usepackage{geometry}
\usepackage{hyperref} % Clickable links
\usepackage{graphicx} % Images

\geometry{a4paper, margin=1in}

\title{Architettura degli elaboratori}
\author{Matteo}
\date{\today}

\begin{document}

\maketitle

\section{Introduction}
Un calcolatore è un sistema composto da processsori, memorie, e dispositivi di input/output.
un \textbf{Bus} è un insieme di connessioni elettriche parallele che trasportano dati da un componente all'altro.

Nell'architettura di Von Neumann viene introdotta l'idea di usare la memoria non solo per i dati ma anche per il programma.

Collega la cpu e la memoria con un bus indirizzi, indicando la posizione dei dati. 
Mentre su un altro bus dati memoria e cpu si scambiano i dati veri e propri.

CPU composta date
\begin{itemize}
\item Unita di controllo: legge e interpreta le istruzioni.
\item Unita aritmetico logica (ALU): esegue operazioni aritmetiche e logiche.
\item registri: memorie veloci interne alla cpu.
\end{itemize}

ci sono diversi registri specializzati:
\begin{itemize}
\item Program counter (PC): indica la prossima istruzione da eseguire.
\item Memory address register (MAR): contiene l'indirizzo di memoria da cui leggere o scrivere dati.
\item Memory data register (MDR): registro che accede ai bus dei dati.
\item Instruction register (IR): contiene l'istruzione che si sta per eseguire.
\item Program status word (PSW): contiene informazioni sullo stato della CPU.
\end{itemize}

in pratica il ciclo di esecuzione di un'istruzione è:
\begin{enumerate}
    \item il contenuto di PC viene copiato in MAR e inviato alla memoria attraverso il bus indirizzi.
    \item la memoria risponde inviando il dato all'MDR attraverso il bus dati.
    \item il contenuto di MDR viene copiato in IR e decodificato.
    \item l'istruzione passa alla ALU per essere eseguita.
    \item se servono altri dati vengono letti dalla memoria nello stesso modo.
    \item se serve il risultato e copiato in memoria attraverso MDR
    \item il PC viene aggiornato per puntare alla prossima istruzione.
\end{enumerate}
 Questo ciclo e chiamato \textbf{fetch-decode-execute cycle}.

 \textbf{data path}: insieme di registri e ALU che eseguono operazioni sui dati.

 il percorso dei dati da memoria ad ALU, la loro esecuzione e il ritorno in memoria viene chiamato \textbf{ciclo di data path}, ed e governato da un clock.

 durata ciclo di data path o ciclo di clock= 1/F, dove F e la frequenza di lavoro della CPU(cicli al secondo), misurata in heartz.

 la velocita di esecuzione delle istruzioni ISA(instruction set architecture), è misurata con la durata di un ciclo di clock per i cicli necessari.

 fissato il ciclo di clock, la velocita puo essere aumentata con il parallelismo.

 negli anni 70 si sviluppo la differenza tra CISC e RISC.
  CISC: complex instruction set computer, set di istruzioni complesso, con istruzioni che fanno operazioni complesse in un singolo ciclo.
  RISC: reduced instruction set computer, set di istruzioni ridotto, con istruzioni semplici che richiedono piu cicli per operazioni complesse.

  Risc usa la microprogrammazione, in modo da far fare diverse operazioni a pochi componenti.\\

  piu cicli FDE possono essere eseguiti insieme con il \textbf{pipelining}

  si sviluppano anche architetture che permettono il parallelismo.
  , ad esempio cpu con piu ALU e control units.

  oppure molte cpu possono lavorare in modo coordinato sulla stessa istruzione su dati diversi, creando un \textbf{array computer}

se invece eseguono istruzioni diverse su dati, con una memoria condivisa, diversi si parla di \textbf{multiprocessor systems}.

la forma piu complessa sono i \textbf{multicomputer systems}, con piu cpu, ognuna con memoria propria, collegate in rete.

\subsection{memoria}

vari tipi di memoria
\begin{itemize}
\item memoria volatile: perde il contenuto quando viene spenta(es:ram).
\item memoria non volatile: mantiene il contenuto quando spenta(es:rom, hard disk).
\item memoria online: sempre accessibile
\item memoria offline: il supporto deve essere montato
\end{itemize}

la memoria si organizza in celle, ogni cella è una sequenza di bit con il proprio indirizzo.

cella di 8 bit=byte.

molti calcolatori lavorano su blocchi da 32 o 64 bit, questi blocchi si chiamano word.

le word possono essere memorizzate in celle standard, occupando piu celle consecutive.
si puo fare in 2 modi:
\begin{itemize}
    \item big-endian: byte piu significativo all'indirizzo piu basso.
    \item little-endian: byte meno significativo all'indirizzo piu basso.
\end{itemize}

la \textbf{cache} è una memoria poco capiente ma  veloce, che viene usata per contenere le word piu usate.

la cpu prima cerca i dati nella cache, se non li trova li va a prendere in memoria principale.

\textbf{principio di località}: dati usati recentemente gli uni dagli altri sono spesso in locazoni vicine.

tipi di supporti:
\begin{itemize}
\item hard disk: memoria non volatile, lenta, grande capacita, usa piatti magnetici rotanti.(puo essere resa piu veloce con la tecnica RAID, che usa piu dischi in parallelo).
\item SSD: memoria non volatile, veloce, grande capacita, usa memoria flash, piu veloce ma meno capiente.
\item CD e DVD: memoria non volatile, usano supporti ottici, capacita limitata.
\end{itemize}

\section{porte logiche e circuiti combinatori}






\end{document}
