\documentclass[12pt]{article}
\usepackage[utf8]{inputenc}
\usepackage{amsmath, amssymb} % Math packages
\usepackage{geometry}
\usepackage{hyperref} % Clickable links
\usepackage{graphicx} % Images

\geometry{a4paper, margin=1in}

\title{tecnologie web}
\author{Matteo}
\date{\today}

\begin{document}


\maketitle
\section{intro}
\section{semantic web}
semantic web:scrtuttura comune che consente di condividere e riutilizzare dati tra applicazioni, aziende e comunita
ciattion link: permette di creare collegamenti
semantic web stack: illustra l'architettura del web semantico
divverese classi sono oragnizzate in modo tassonomico attraverso un modello chiamato RDF.

RDF(resource description framework):serve a fare affermazioni(statement) sulle risorse nella forma di triple(soggetto-predicato-oggetto)

un garfo RDF e un insieme di triple RDFle risorse sono rappresentate come nodi.

il modello a triple e semplice e minimalista. il modello RDF inoltre e modulare:
\begin{itemize}
    \item la gesione delle informazioni puo essere parallelizzata
    \item informazioni parziali sono comunque valide
    \end{itemize}

    svantaggi:
    \begin{itemize}
        \item il modello di dati RDF e costituito da elementi di dati piccoli e frammentati, quindi un database di medie dimensioni risulta in miliardi di triple
        \item limitazione delle relazioni n-arie, non ci sono modi semplici di descriverli
        \item limitata possibilita di attribuire informazioni alle triple stesse
    \end{itemize}

    reificazione: prendo una tripla e gli do un identificativo per farla diventare parte di un alòtra tripla


    microformati: sono ad esempio embedding di triple RDF all'interno di ambienti ospiti

    importante l'aspetto della serializzazione.
    manca il reasoning  e la generazione di nuove informazioni

    con RDF si possono introdurre le inferenze attraverso altre cllassi eproprieta

    il modello tabellare e il migliore che abbiamo nel complesso, ma e inefficente per le relazioni M-N
    il modello ad albero ha senso per le relazioni di tipo 1-N meno mer M-N

    


\end{document}